\documentclass{article}
\usepackage[spanish]{babel}
\usepackage[utf8]{inputenc}
\usepackage[T1]{fontenc}
\usepackage{vmargin}
\usepackage{setspace}
\usepackage{enumerate}
\usepackage{graphicx}
\graphicspath{ {images/} }
\usepackage{float} 

\begin{document}
\begin{center}
\includegraphics[scale=0.5]{index.jpeg}
\\
\vspace{0.5cm}
UNIVERSIDAD DE SONORA \\
\vspace{0.5cm}
DIVISIÓN DE CIENCIAS EXACTAS Y NATURALES \\
\vspace{0.5cm}
DEPARTAMENTO DE FÍSICA\\
\vspace{0.5cm}
LICENCIATURA EN FÍSICA\\
\vspace{0.5cm}
FÍSICA COMPUTACIONAL I

\vspace{2 cm}
\hrule
\vspace{1 cm}

{\huge \bfseries {La Atmósfera Terrestre}}
\\

\vspace{1 cm}
\hrule
\vspace{2 cm}
Ricardo Ruiz Hernández\\ 
\vspace{1 cm}
Profesor del curso\\
Dr. Carlos Lizárraga Celaya\\
\vspace{2 cm}
30 de Enero del 2018
\end{center}

\pagebreak
\begin{doublespace}

\hrule
\section*{Resumen}
A continuación describiré, de manera general, lo que la atmósfera terrestre es, enfocando nuestra atención principalmente en sus capas; no obstante, hablaremos también de distintas propiedades que posee la atmósfera terrestre, para ello revisaremos sitios web que nos facilitarán información de mucha utilidad, puesto que las referencias son siempre importantes.
\vspace{0.5 cm}
\hrule
\vspace{0.6 cm}
\section{Introducción}
Sin dudar, puedo decir que la atmósfera es considerada por muchos uno de los elementos de mayor belleza de la naturaleza, porque, ¿quién no ha pasado una tarde mirando hacia el cielo? solo eso, el cielo y uno; la increíble variedad cromática e infinidad de formas nubosas hacen que ese momento sea único e incluso indescriptible.
\\
Aunque se podrían escribir muchas páginas sobre la belleza de la atmósfera, en esta ocasión, como mencioné antes, describiremos que la compone, abordando todo desde un punto de vista netamente científico. Para esto recurriremos a información proporcionada por las ciencias de la atmósfera, (término genérico para las ciencias que estudian la atmósfera, sus procesos, los efectos que otros sistemas tienen sobre la atmósfera, y los efectos de la atmósfera en estos sistemas). Las ciencias de la atmósfera se ha ampliado a la esfera de la ciencia planetaria y el estudio de las atmósferas de los planetas del sistema solar. Comprende las siguientes disciplinas:
\begin{itemize}
    \item La Meteorología incluye la química atmosférica y física atmosférica, con especial énfasis en la predicción del tiempo.
    \item La Climatología es el estudio de los cambios atmosféricos (tanto a largo como a corto plazo) que definen los climas promedio y su cambio en el tiempo, debido tanto a la variabilidad natural como antropogénica.
    \item	La Aeronomía es el estudio de las capas superiores de la atmósfera, donde la disociación y la ionización son importantes.    
\end{itemize}

Bien, procederé a definir atmósfera, que, aunque nos parezca algo trivial, y que ya todos conocemos, será necesario ampliar nuestra definición sobre ella para caminar de manera más sencilla por este trabajo de investigación.

\section{Definición y características generales}
La palabra atmósfera es un término compuesto por dos partículas, atomos, que en griego significa vapor, aire y la palabra esfera. Es decir que es la envoltura gaseosa que cubre a una esfera o cuerpo celeste o a un planeta. Cada cuerpo celeste tiene una atmósfera propia, de características particulares. En el caso de la Tierra, la atmósfera seca, o sea sin considerar el vapor de agua presente en ella, está compuesta por:
\begin{table}[h]
\centering
\label{my-label}
\begin{tabular}{|l|l|l|ll}
\cline{1-3}
Nitrógeno           & $N_2$  & 78.08\% en volumen &  &  \\ \cline{1-3}
Oxígeno             & $O_2$  & 20.95\% en volumen &  &  \\ \cline{1-3}
Argón               & Ar     & 0.93\% en volumen  &  &  \\ \cline{1-3}
Anhíbrido Carbónico & $CO_2$ & 0.03\% en volumen  &  &  \\ \cline{1-3}
\end{tabular}
\end{table}

El aproximadamente 0.01\% restante esta compuesto por el ozono (0.000006\%) y gran variedad de gases presentes en cantidades muy reducidas, de allí su denominación de gases traza. El vapor de agua puede llegar a ser el 4\% de la atmósfera cerca de la superficie del planeta pero por encima de los 10-15 km solo se encuentra en cantidades muy pequeñas, haciendo parte de ese 0,01 por ciento restante. A pesar de estar presentes en cantidades tan bajas, estos gases tienen una participación muy importante en el comportamiento del clima y el desarrollo de los procesos atmosféricos, debido a su participación en la física y la química que regulan el estado de la atmósfera.

El ozono y el vapor de agua, ambos gases de invernadero, por absorber la radiación infrarroja terrestre, son particularmente importantes para definir el clima terrestre y mantener las condiciones ambientales que permiten la vida en la Tierra. Ello es debido, particularmente, por el papel que juegan sus respectivas distribuciones verticales y su variabilidad geográfica y temporal.


\section{Estructura de la atmósfera}
En función del comportamiento de la temperatura atmosférica con la altura, convencionalmente la atmósfera terrestre ha sido dividida en cinco capas que, a partir de la superficie terrestre, se denominan consecutivamente: Tropósfera, Estratósfera, Mesósfera, Termósfera y Exósfera.

\subsection{Tropósfera}
La tropósfera es la capa más cercana a la superficie de la Tierra. Tiene de 4 a 12 millas (7 a 20 km) de espesor y contiene la mitad de la atmósfera de la Tierra. Esta parte de la atmósfera es la más densa, pues aquí se concentra la mayor parte del oxígeno y vapor de agua. El vapor de agua permite regular la temperatura. 
\\
Aquí hay variaciones de la temperatura, mismas que son consecuencia del aumento de la altitud, mientras más grande sea esta la temperatura disminuye, ¿la razón?, simple, la topósfera se calienta principalmente a través de la transferencia de energía de la superficie; así pues, la temperatura decrece en un aproximado de 6.5 $^{\circ}$C por cada kilómetro que se asciende.  La troposfera es más densa que todas las capas atmosféricas que la recubren porque un peso atmosférico más grande se encuentra en la parte superior de la troposfera y hace que se comprima más severamente. El cincuenta por ciento de la masa total de la atmósfera se encuentra en los 5,6 km (18,000 pies) más bajos de la troposfera.
\\
Algo que es realmente importante para destacar es el hecho de que casi todo el vapor de agua atmosférico o humedad se encuentra en la troposfera, por lo que es la capa donde tiene lugar la mayor parte del clima de la Tierra, se llevan a cabo todos los fenómenos meteorológicos, como la lluvia, huracanes o vientos.

\subsection{Estratósfera}
La estratósfera se encuentra desde el final de la tropósfera hasta los 50 km de altura, aproximadamente. Contiene la capa de ozono, que es la parte de la atmósfera de la Tierra que contiene concentraciones relativamente altas de ese gas. En esta capa la temperatura aumenta a más altitud, esto se debe a la absorción de los rayos ultravioleta y la presencia de ozono. Este perfil de temperaturas permite que la capa sea muy estable y evita turbulencias, el aire es muy seco y mil veces más delgado aquí que a nivel del mar. Por esta razón, es donde muchos aviones de reacción prefieren volar, así como los globos meteorológicos.

\subsection{Mesósfera}

La mesósfera se extiene hasta los 85 km de altura, aproximadamente. Esta capa contiene la zona más fría de la Tierra, esta se situa en la parte superior, descendiendo la temperatura hasta los -90 $^{\circ}$C. 
\\
La mesósfera es la capa donde la mayoría de los meteoros se queman al entrar en la atmósfera. Al estar demasiado elevada sobre la Tierra, es inaccesible para aviones y globos propulsados por aviones a reacción y, a su vez demasiado bajo para permitir naves espaciales orbitales; sin embargo, la mesósfera si es accesible, principalmente por cohetes que suenan y aviones propulsados por cohetes.

\subsection{Termósfera}
La termósfera se extiende hasta entre los 500 y 1,000 km de altura, aquí la temperatura aumenta gradualmente con la altura, que en contraste con la estratósfera, donde ocurre una inversión de temperatura debido a la absorción de radiación por el ozono, la inversión en la termosfera ocurre debido a la extremadamente baja densidad de sus moléculas, elevándose de esta manera la temperatura hasta 1500 $^{\circ}$C.
\\
Importante mencionar que esta capa se encuentra completamente despejada y libre de vapor de agua. No obstante, ocurren fenómenos no hidrometeorológicos como la aurora boreal y la aurora austral. Cabe destacar tambíén que 
La Estación Espacial Internacional orbita en esta capa, entre 350 y 420 km de altura.

\subsection{Exosfera}
La exosfera es la última capa de la atmósfera terrestre, la cual se extiende hasta los 10,000 km de altura, aproximadamente, lugar donde se funde con el viento solar. La exósfera está formada principalmente por helio e hidrógeno, cuyos átomos pueden alcanzar velocidades suficientemente elevadas como para escapar del campo gravitatorio terrestre, en esta capa hay un alto contenido de polvo cósmico. La exosfera contiene la mayoría de los satélites que orbitan alrededor de la Tierra.
\begin{center}
\includegraphics[scale=0.5]{Limite.jpg}
\end{center}

\section{¿Cómo se estudia la atmósfera?}
Hasta este punto ya se mencionaron, en términos generales, algunas de las características de nuestra atmósfera, así como descripción de las capas que la componen; ahora bien, es justo que nos preguntemos, ¿cómo se obtuvo esta información?, bien, aunado a las arduas investigaciones a lo largo de la historia para predecir el tiempo y describir cada vez con más detalle lo que protege a nuestro planeta, surgió una herramienta crucial, llamada globo metereológico, el cual supuso un antes y un después en las ciencias que investigan estos fenómenos. 
\\
Una de las primeras personas en usar el globo meteorológico, fue el meteorólogo francés Léon Teisserenc de Bort, que a partir de 1896 lanzó cientos de globos meteorológicos desde su observatorio en Trappes, Francia, esto llevó al descubrimiento de la tropopausa y la estratosfera. El globo meteorológico, es un globo aerostático compuesto de látex de gran flexibilidad. Este globo eleva instrumentos a la atmósfera para obtener información sobre la presión, humedad, temperatura y velocidad del viento en las zonas de la atmósfera, esta información es suministrada por medio de un aparato pequeño de medición llamado radiosonda, dicho aparato es desechable. Para obtener los datos del viento, pueden ser rastreados por radar, radiolocalización o bien por sistemas de navegación (como el GPS).

\begin{center}
\includegraphics[scale=0.5]{globo.jpg}
\end{center}

\section{Conclusión}

Nuestra atmósfera es de vital importancia para todos, pues gracias a la atmósfera la vida se puede desarrollar en nuestro planeta, ya que absorbe gran parte de la radiación ultravioleta del sol en la capa de ozono. Otra función importante que tiene la atmósfera es la magnetosfera. Ésta es una zona de la atmósfera que se encuentra en la región exterior de la Tierra que nos protege desviando los vientos solares cargados de radiación electromagnética. Es gracias al campo magnético terrestre que no somos consumidos por las tormentas solares. 
\\
También hay que mencionar el hecho de que la atmósfera terrestre no ha tenido siempre la misma composición. Durante millones de años, la composición de la atmósfera ha ido cambiando y generando otro tipos de vida. Por ejemplo, cuando la atmósfera apenas tenía oxígeno, era el gas metano quien regulaba el clima y la vida que predominaba era la de los metanógenos. Tras la aparición de las cianobacterias, aumentó la cantidad de oxígeno en la atmósfera e hizo posibles distintas formas de vida como son las plantas, los animales y los seres humanos.
\\
La atmósfera tiene gran relevancia en el desarrollo de los ciclos biogeoquímicos. La composición actual de la atmósfera se debe a la fotosíntesis realizada por las plantas. También es la que controla el clima y el ambiente en el que vivimos los seres humanos (en la troposfera) generando los fenómenos meteorológicos como la lluvia (de la que conseguimos el agua) y teniendo la concentración de nitrógeno, carbono y oxígeno necesarios.
\\
Así pues, esta gran burbuja que rodea a nuestro planeta es realmente plurifuncionak, lo que la hace un objeto muy interesante de estudio. Sin lugar a dudas, las ciencias atmosféricas seguirán creciendo y arrojando cada vez más información referente a lo que nos mantiene con vida en el catastrófico universo, la atmósfera. 

\section*{Referencias}
\begin{itemize}
    \item https://en.wikipedia.org/wiki/Atmosphere Consultada el 28 de Enero del 2018.
    \item http://climate.ncsu.edu/edu/k12/.AtmStructure Consultada el 30 de Enero del 2018.
    \item $http://www.cricyt.edu.ar/enciclopedia/terminos/Atmosfera.htm$ Consultada el 31 de Enero del 2018.
    \item $http://www.windows2universe.org/earth/Atmosphere/overview.html1$ Consultada el 01 de Febrero del 2018.
    \item Imagen 2. $https://c2.staticflickr.com/8/7007/13608112015_eed7f2273a_b.jpg$ Consultada el 02 de Febrero del 2018.
    \item $https://en.wikipedia.org/wiki/Weather_balloon$ Consultada el 02 de Febrero del 2018.
    \item Imagen 1. $https://c2.staticflickr.com/8/7007/13608112015_eed7f2273a_b.jpg$ Consultada el 01 de Febrero de 2018
  
    
\end{itemize}

\end{doublespace}
\end{document}

