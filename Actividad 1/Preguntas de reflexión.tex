\documentclass{article}
\usepackage[spanish]{babel}
\usepackage[utf8]{inputenc}
\usepackage[T1]{fontenc}
\usepackage{vmargin}
\usepackage{setspace}
\usepackage{enumerate}
\usepackage{graphicx}
\graphicspath{ {images/} }
\usepackage{float} 

\begin{document}
\begin{center}
\includegraphics[scale=0.5]{unison1.jpg}
\\
\vspace{0.5cm}
UNIVERSIDAD DE SONORA \\
\vspace{0.5cm}
DIVISIÓN DE CIENCIAS EXACTAS Y NATURALES \\
\vspace{0.5cm}
DEPARTAMENTO DE FÍSICA\\
\vspace{0.5cm}
LICENCIATURA EN FÍSICA\\
\vspace{0.5cm}
FÍSICA COMPUTACIONAL I

\vspace{2 cm}
\hrule
\vspace{1 cm}

{\huge \bfseries {Preguntas de reflexión}}
\\

\vspace{1 cm}
\hrule
\vspace{2 cm}
Ricardo Ruiz Hernández\\ 
\vspace{1 cm}
Profesor del curso\\
Dr. Carlos Lizárraga Celaya\\
\vspace{2 cm}
30 de Enero del 2018
\end{center}

\pagebreak
\begin{doublespace}

\begin{center}
\section*{Preguntas de reflexión}
\end{center}

\begin{itemize}
{\item \bfseries ¿Qué fue lo que más te llamó la atención de esta actividad?}
\\
Sin duda la utilización de LATEX, puesto que dejamos de lado lo tradicional y damos un paso más editando nuestros textos de manera más profesional. Además, de que la atmósfera siempre ha sido algo que llama mi atención, por lo tanto, me quedo con estas dos cosas, el aprendizaje de ambas cuestiones.

{\item \bfseries ¿Qué fue lo que se te hizo menos interesante?}
\\
Francamente todo me pareció interesante, disfruté realizando este trabajo de investigación.

{\item \bfseries ¿Qué cambios harías para mejorar esta actividad?}
\\
Aunque me gusta tener libertad a lo hora de plasmar una investigación, me parece que es importante que nos adaptemos a un estilo de trabajo en concreto (dependiendo de la situación), me refiero a que sería interesante y útil se nos pidiese un tipo de texto (monografía, ensayo, etcétera) y así conocer las distintas formas que pueden tener nuestras investigaciones.

{\item \bfseries ¿Cuál es tu primera impresión de uso de LATEX?}
\\
Es muy interesante y amplio. Me llevo la impresión de que hay mucho terreno por abarcar, lo cual me deja de cierto modo satisfecho, puesto que mis trabajos pueden mejorar mucho aun, en cuanto a calidad de presentación se refiere.

{\item \bfseries ¿El tiempo sugerido para esta actividad fue suficiente?}
\\
Me parece que sí, a pesar de que por motivos personales no logré terminar dentro de los límites, el tiempo que empleé realizándola corresponde al que se brindó en un principio.

{\item \bfseries ¿Encontraste algún documento o recurso en línea útil que quisieras compartir con los demás?}
\\
Además de los sitios de internet que ya plasmé en las referencias del trabajo, encontré una herramienta que en esta ocasión me fue útil: $https://www.tablesgenerator.com$; en esta página generé la tabla de datos que se encuentra dentro de mi actividad.

\end{itemize}
\end{doublespace}
\end{document}
